\documentclass{scrartcl}
\usepackage[utf8]{inputenc}
\usepackage[T1]{fontenc}
\usepackage{graphicx}
\usepackage{grffile}
\usepackage{longtable}
\usepackage{wrapfig}
\usepackage{rotating}
\usepackage[normalem]{ulem}
\usepackage{amsmath}
\usepackage{textcomp}
\usepackage{amssymb}
\usepackage{capt-of}
\usepackage{hyperref}
\usepackage[main=british, polutonikogreek, english]{babel}

\title{Task Instructions}
\author{AD Bolton}
\date{January 2021}

\begin{document}

\maketitle

\begin{abstract}

  Hello and thanks for participating! The test you are about to take examines the human ability to distinguish fundamental types of motion described in physics and everyday life. The task also addresses whether humans can explicitly combine simple motion types into more complex percepts. This project is largely motivated by the work of Gunnar Johansson, who showed that only a few white dots in motion on a black background can induce instantaneous perceptions of complex human activities like walking, dancing, and exercising. Our project seeks to examine the underpinnings of how this type of `biomotion` is generated and perceived. 

  As a subject, you will be presented 30 patterns of moving white dots. Your first job is to distinguish the motion type of each dot in the scene (Periodic, Brownian, Uniform Linear, Accelerating Linear).

\begin{enumerate}
  \item Periodic motion means that the same pattern of motion repeats multiple times during the stimulus.
  \item Brownian motion implies pure randomness -- the dot will move with no particular pattern in random directions as the stimulus unfolds.
  \item Uniform Linear motion means the dot is moving in a straight line at a constant speed.
  \item Accelerating Linear motion implies that the dot is also moving in a straight line, but speeding up. 
\end{enumerate}

  Next, on some trials, a subset of dots in the scene, in addition to its own motion, will have inherited the motion of another dot. It is your job to decide which dots have inherited the motion of which other dots. Dots that influence other dots can be thought of as ``parents'' and dots that receive influence from other dots are ``children'' in a scene graph.

\end{abstract}

\vspace{10mm}

\section{Task Logistics}

When you are ready to begin the task, simply run the `run_human_experiment()` command in the provided REPL. A window will launch showing one to three stationary dots with labels 1-3. You are free to position this window wherever you choose. After a few seconds the dots will begin to move. This stimulus will repeat as many times as you'd like. When you are ready to answer to assign motion types and groupings to each labeled dot, press enter. The window will change to an answer portal where the motion type of each dot should be selected from a drop-down menu. If you believe that a particular dot in the scene is influenced by another dot, switch the toggles that indicate your perceptions. Below the toggles are two sliders: one indicates the amount of confidence in the accuracy of your selection, while the other indicates how similarly you feel the stimulus was to biological motion. 
  

\end{document} 

  
%%% Local Variables:
%%% mode: latex
%%% TeX-master: t
%%% End:
