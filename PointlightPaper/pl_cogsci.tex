
%%% Local Variables:
%%% mode: latex
%%% TeX-master: t
%%% reftex-default-bibliography: ("./pl_cogsci.bib")
%%% End:

% koma / GEB appearance

\documentclass{scrartcl}
\usepackage[utf8]{inputenc}
\usepackage[T1]{fontenc}
\usepackage{graphicx}
\usepackage{grffile}
\usepackage{longtable}
\usepackage{wrapfig}
\usepackage{rotating}
\usepackage[normalem]{ulem}
\usepackage{amsmath}
\usepackage{textcomp}
\usepackage{amssymb}
\usepackage{capt-of}
\usepackage{hyperref}
\usepackage[main=british, polutonikogreek, english]{babel}

\title{A Gaussian Process Model of Pointlight Perception}
\author{AD Bolton, FA Saad, MF Cusumano-Towner, VK Mansinghka}
\date{January 2021}

\begin{document}

\maketitle

\begin{abstract}
  Changes in the intensity, qualities, or position of a stimulus over time can significantly affect how humans interpret its meaning. One example of this comes from Gunnar Johansson's work on ``Point Light Stimuli'', which revealed that ambiguous arrangements of only a few stationary dots can induce striking perceptions of humans when set in biologically realistic motion. Here, we seek to further elucidate how perceptual meaning arises from the integration of fundamental motion cues. Specifically, we compose dynamic dot scenes by hierarchical combination of four standard motion types observed in physics and everyday life. Using the probabilistic programming system Gen, we implement a Gaussian Process based generative model that first assigns a motion type to each dot (Brownian, Periodic, Uniform Linear, or Accelerating Linear), then generates a directed scene graph that describes inheritance of motion from one dot to another. In this way, ``groups'' of dots are specified that share common motion features and suggest unified objects. The goal of the task we devise is to infer the motion type of each dot and decide which dots belong to which groups (i.e. infer the scene graph); we posit that this constructive approach mirrors the cognitive processes that humans use to identify objects and meaning in sparse motion patterns. Lastly, we implement importance resampling inference constrained on only the observation of position and motion of the dots. We show that our inference program uncovers the correct scene graph and motion type assignment almost 100\% of the time, and that uncertainty in the inference program's posterior reflects the perceptual ambiguity experienced by human observers. However, there are common exceptions where inheritance of motion from one dot to another does not lead to grouping, and other examples where inheritance assignments are surprisingly ambiguous. We address these results below. Overall, we propose that the Gaussian Process abstraction we have implemented enables accurate and explicit reasoning about dot motion scenes, providing a framework for explaining percepts that emerge from random dot patterns.
\end{abstract}

\vspace{10mm}



A set of phenomena observed in past literature and in our own experiences with our model are shown below. Experiments to address questions raised by each phenomenon follow.

\section{Phenomena} \label{Phenomena}

\begin{enumerate}
  \item \label{motion_alone_phen}
  Physicists studying motion (e.g. Brown, Hooke) and humans in their colloquial language (``rhythmic'', ``bobbing'', ``meandering'', ``speeding up'', ``slowing down'') appear to be able to distinguish distinct types of motion in the physical world. However, no studies have posited the generative process that mediates this cognitive ability, nor whether humans can consistently reason about complex motion trajectories in terms of simpler fundamental types. E.g. A sailor's complex trajectory while walking across a moving sailboat is composed of the boat's uniform forward translation, his / her own walking path, the random fluctuations of the wind, and the periodic rhythm of waves. 
  \item \label{biomotion_phen} A sparse set of white dots moving on a black background can create percepts of \emph{biological motion} (e.g. humans walking or dancing, fish swimming) \cite{Johansson_1973}. No studies have addressed how primitive forms of motion are composed to create biological motion. 
  \item Dots that share common vectors of motion lead to perceptual grouping (i.e. ``Theory of Vector Analysis'') \cite{Johansson_1973}. In this way, common motion of a set of dots relative to its background allows the recognition of unified objects. However, it is unknown whether this grouping applies to all motion types and whether there are special cases where objects that share motion vectors are not perceived as groups. \label{common_motion_phen}. Our preliminary results indicate that inheritance between two dots moving linearly creates ambiguity in scene graph interpretation. Moreover, periodic motion inheritance often creates ambiguity in object / reference frame distinctions. We will examine these phenomena further in our human study. 
\end{enumerate}


\section{Experiments}

Human subjects will be presented a ten second movie of 1, 2, or 3 white dots moving on a black background. (Here, cite a figure that describes math and code that generates the stimulus). The subject will be allowed to replay the stimulus until selecting an answer (described below). In each case, the distribution of human results will be compared to inference results obtained from running importance resampling constrained on velocity and position of the dots in the observed stimulus. 

Phenomena described in section \ref{Phenomena} will be addressed with the following experiments.

\begin{enumerate}
  \item After observing the 10 second video, the goal of the subject is to identify the motion type of each dot (Uniform Linear, Accelerating Linear, Periodic, Brownian), and decide which dots are grouped (i.e. infer the scene graph). The 1-dot case queries the subject's ability to identify the common patterns of motion that we suggest are fundamental building blocks of complex motion. 
  \item On each trial, the subject will be required to enter a value indicating, on a scale of 1-10, how resemblant the stimulus was to biological motion. This will allow us to assess how biological motion is composed from primitive motion types. 
  \item On each trial, subjects will be asked to judge their confidence that a particular velocity profile and grouping is correct. The confidence level will be compared to the inference posterior on grouping vs. ungrouping to judge how well our model captures uncertainty in scene interpretation.
\end{enumerate}

\section{Metrics and Figures}
\begin{enumerate}
  \item Figure 1: Drawn out sailor walking on a boat example. Show complexity of the actual trajectory of the sailor walking, then decomposition into common motion types (Brownian, Uniform Linear, Periodic).
  \item Figure 2: Plot accuracy of scene graph inference vs. number of importance particles for 1, 2, and 3 dot sets. Plot probability of most likely graph vs. number of particles as a metric for uncertainty.
  \item Figure 3: After choosing the correct \# of particles, compare accuracy of inference program to human data. Compare uncertainty responses of subjects to probability of top graph, and also KL-divergence between population responses and full posteriors. Likely here to see grouping errors in linear-linear inheritance graphs. If you do, make a separate panel for this effect. 
  \item Figure 4: Plot top 5 scene graphs for biomotion metric. Plot bottom 5 scene graphs. Could also rank in a supplementary heat map for all 2 and 3 dot patterns. 
\end{enumerate}
  
\section{Contributions}

\begin{enumerate}
  \item A Gaussian Process-based cognitive model that constructs dynamic dot motion stimuli using a prior over common motion types and scene graphs. \cite{Saad_2019}
  \item Gen-based importance resampling results showing that accurate inference of scene graph and motion type structure is attainable within the context of our model. 
  \item Experimental data on human subjects suggesting that the model and inference accomplished by our program may mirror human cognitive processes.
  \item Explicit recipes for generating random stimuli that at times resemble the features of bio-motion.
\end{enumerate}    

\section{Claims}

\begin{enumerate}
  \item Non-parametric Bayesian Models (i.e. CRP) defining flow fields can recapitulate (1). These careful studies explained the phenomenological effects in Johansson's studies, but fail to posit the actual latent structure that exists in humans' perceptual machinery \cite{Gershman_2016}. Our model posits the latent structure that generates motion timeseries and captures human abilities in motion type identification and grouping inference tasks.
  \item We do not believe the Vector Analysis method works for the case of linear-linear inheritance in scene graphs nor in the ability of humans to distinguish figure / background in periodic->linear inheritance.
  \item It is likely that the explicit composition of primitive motion types in this study will lead to a better understanding of the characteristics of stimuli that create biomotion percepts. 
\end{enumerate}

\section{Future Directions}

Our current program requires memory of 2-3 entire timeseries of motion to make perceptual judgements (i.e. the full time series is used to condition importance sampling). It is therefore incapable of revealing fluctuations in certainty about stimulus structure over time. It is also possible that heuristics are involved in defining motion types that could be used as custom proposals to inference algorithms. Future studies will add dots, motion types, and other common GP kernels in order to achieve more complex types of motion that will likely call for custom SMC inference algorithms. 

\bibliographystyle{vancouver}
\bibliography{pl_cogsci.bib}

\end{document}
