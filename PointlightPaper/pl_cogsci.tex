
%%% Local Variables:
%%% mode: latex
%%% TeX-master: t
%%% End:

\documentclass[12pt, letterpaper]{article}


\usepackage[utf8]{inputenc}
\usepackage{graphicx}
\title{A Gaussian Process Model of Pointlight Perception}
\author{Andrew D Bolton, Vikash K Mansinghka}
\date{January 2021}


\begin{document}

\maketitle

\begin{abstract}
Changes in the intensity, qualities, or position of a stimulus over time can significantly affect how humans interpret its meaning. One example of this comes from the pioneering work of Gunnar Johansson on ``Point Light Stimuli''. This work showed that ambiguous stationary dot patterns induce striking perceptions of humans when set in bio-realistic motion. Here, we address how this sort of deep perceptual meaning arises from the most fundamental hints of its structure. Specifically, we compose dynamic dot scenes by hierarchical combination of four standard motion types studied in physics and observed in every day life. Using the probabilistic programming system Gen, we implement a GP-based generative model that first assigns a motion type to each dot (Brownian, Periodic, Uniform Linear, or Accelerating Linear), then generates a directed scene graph that describes inheritance of motion from one dot to another. In this way, ``groups'' of dots are specified that share motion features, and an inference task becomes inferring which dots belong to which groups. We use importance resampling inference constrained on only the observation of position and motion of the dots, and reveal that our inference program can uncover the correct scene graph and motion type assignment almost 100\% of the time. Moreover, the uncertainty revealed during inference reflects the uncertainty experienced by human observers characterizing the motion types and inheritance patterns in the stimuli. 


\section*{Contributions}

\begin{enumerate}
  \item A Gaussian Process-based cognitive model that constructs dyanamic dot motion stimuli using a prior over common motion types and scene graphs.
  \item Gen-based importance resampling results showing that accurate inference of scene graph and motion type structure is possible
  \item Experimental data on human subjects shows that the inference patterns accomplished by our program mirror human choices. 
\end{enumerate}    

\section*{Claims}

\begin{enumerate}
  \item Our model is the first to posit the latent structure that generates motion timeseries and captures human abilities in motion type and grouping inference tasks
  \item Using GP-generated motion should allow translation to neural network architectures that are closer to biophysically realistic models for human visual perception
  \item Explicit symbolic descriptions of motion patterns in our model reflect humans' ability to express knowledge of different motion types and combinations
  \item Combining simple well-known motion types yields interesting percepts of motion that resemble biological stimuli
  \item Studying the fundamental elements of motion perception should lead to testable hypothesis in real neural circuits known to encode moving dots
\end{enumerate}


\section*{Phenomena}
  
\begin{enumerate}
  \item Shared motion vectors of dots in previous studies have contributed to grouping (i.e. ``Theory of Vector Analysis'')
  \item Non-parametric Bayesian Models (i.e. CRP) defining flow fields can recapitulate (1) but fail to posit latent structure
  \item Physicists studying motion (e.g. Brown, Hooke) and humans in their colloquial language (``bouncing'', ``bobbing'', ``speeding up'', ``slowing down'') are clearly able to distinguish distinct motion types in the physical world. 
  \item Uncertainty arises in the groupings of dots that could be explained if multiple scene graphs showed relatively high posterior probability in our model. 
\end{enumerate}

\section*{Experiments}



  
\end{abstract}

\end{document}
