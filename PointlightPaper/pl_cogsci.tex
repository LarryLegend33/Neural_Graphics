
%%% Local Variables:
%%% mode: latex
%%% TeX-master: t
%%% reftex-default-bibliography: ("./pl_cogsci.bib")
%%% End:

% koma look

\documentclass{scrartcl}
\usepackage[utf8]{inputenc}
\usepackage[T1]{fontenc}
\usepackage{graphicx}
\usepackage{grffile}
\usepackage{longtable}
\usepackage{wrapfig}
\usepackage{rotating}
\usepackage[normalem]{ulem}
\usepackage{amsmath}
\usepackage{textcomp}
\usepackage{amssymb}
\usepackage{capt-of}
\usepackage{hyperref}
\usepackage[main=british, polutonikogreek, english]{babel}


% \documentclass[12pt, letterpaper]{article}
% \usepackage[utf8]{inputenc}
% \usepackage{graphicx}

\title{A Gaussian Process Model of Pointlight Perception}
\author{Andrew D Bolton, Feras A Saad, Marco F Cusumano-Towner, Vikash K Mansinghka}
\date{January 2021}

\begin{document}

\maketitle

\begin{abstract}
  Changes in the intensity, qualities, or position of a stimulus over time can significantly affect how humans interpret its meaning. One example of this comes from Gunnar Johansson's work on ``Point Light Stimuli'', which revealed that ambiguous arrangements of stationary dots can induce striking perceptions of humans when set in biologically realistic motion. Here, we address how this type of deep perceptual meaning could arise from the integration of fundamental motion cues. Specifically, we compose dynamic dot scenes by hierarchical combination of four standard motion types observed in physics and everyday life. Using the probabilistic programming system Gen, we implement a GP-based generative model that first assigns a motion type to each dot (Brownian, Periodic, Uniform Linear, or Accelerating Linear), then generates a directed scene graph that describes inheritance of motion from one dot to another. In this way, ``groups'' of dots are specified that share common motion features and suggest unified objects. The goal of the task we devise is to infer the motion type of each dot and decide which dots belong to which groups; we posit that this constructive approach mirrors the cognitive processes that humans use to identify objects and meaning in sparse motion patterns. Lastly, we use importance resampling inference constrained on only the observation of position and motion of the dots, and show that our inference program uncovers the correct scene graph and motion type assignment almost 100\% of the time. Moreover, uncertainty in the inference program's identification of our stimuli reflects the uncertainty experienced by human observers characterizing the motion types and inheritance patterns in the stimuli.

  Overall, the Gaussian Process abstraction enables explicit reasoning about dot motion scenes: our model is useful because when unexpected percepts pop out of random dot patterns (e.g. ``a jumping frog'', ``a swimming fish''), the explicit nature of the model's generative process leads to clear answers about how these percepts are composed. This idea, and future directions incorporating more motion types and scene graph arrangements, will be addressed below. 

\end{abstract}

\section*{Contributions}

\begin{enumerate}
  \item A Gaussian Process-based cognitive model that constructs dynamic dot motion stimuli using a prior over common motion types and scene graphs. \cite{Saad_2019}
  \item Gen-based importance resampling results showing that accurate inference of scene graph and motion type structure is attainable within the context of our model. 
  \item Experimental data on human subjects suggesting that the model and inference accomplished by our program may mirror human cognitive processes.
  \item Explicit recipes for generating random stimuli that at times resemble the features of bio-motion. 

\end{enumerate}    

\section*{Claims}

\begin{enumerate}
  \item Our model posits the latent structure that generates motion timeseries and captures human abilities in motion type and grouping inference tasks.
  \item Explicit symbolic descriptions of motion patterns in our model reflect humans' ability to express knowledge and reason about different motion types and motion combinations.
  \item Studying the fundamental elements of motion perception using dot stimuli should lead to testable hypothesis in real neural circuits known to encode the position and velocity of moving dots. \cite{Bolton_2019}
  \item Our use of GP-generated motion should allow translation to neural network architectures that have been found to be equivalent to GPs, which may provide inroads to visual neuroscience. \cite{Neal_1996}
\end{enumerate}


\section*{Phenomena}
  
\begin{enumerate}
  \item Shared motion vectors of dots in previous studies have contributed to grouping (i.e. ``Theory of Vector Analysis'') \cite{Johansson_1973}
  \item Non-parametric Bayesian Models (i.e. CRP) defining flow fields can recapitulate (1) but fail to posit latent structure \cite{Gershman_2016}
  \item Physicists studying motion (e.g. Brown, Hooke) and humans in their colloquial language (``bouncing'', ``bobbing'', ``meandering'', ``speeding up'', ``slowing down'') are clearly able to distinguish distinct types of motion in the physical world. 
  \item Uncertainty arises in the groupings of dots that could be explained if multiple scene graphs showed relatively high posterior probability in our model. 
\end{enumerate}

\section*{Experiments}

Humans will be presented a ten second movie of 2 or 3 white dots moving on a black background. These dot patterns will be generated directly from our model. The subject will be allowed to replay the stimulus at will until selecting an answer. The goal of the subject is to identify the motion type of each dot and decide which dots are grouped. Human results will be compared to inference results obtained from running importance resampling constrained on velocity and position of the dots using KL divergence. 5-10 subjects in the first pass should be sufficient for identifying whether the task is viable. By Neurips, may be beneficial to have more human data and zebrafish data showing differential responses to groups vs. non-groups. 

\section*{Future Directions}

Our current program requires memory of 2-3 entire timeseries of motion to make perceptual judgements (i.e. the full time series is used to condition importance sampling). It is therefore incapable of revealing fluctuations in certainty about stimulus structure over time. It is also possible that heuristics are involved in defining motion types that could be used as custom proposals to inference algorithms. Moreover, in this work, we have constrained the parameters of the GPs' covariance functions to fixed values. Future studies will add dots, motion types, and GP hyperparameters in order to achieve more complex types of motion that will likely call for custom SMC inference algorithms. 

\bibliographystyle{vancouver}
\bibliography{pl_cogsci.bib}

\end{document}
